\documentclass[BSc]{abdnthesis}

%% For citations, I would recommend natbib for its
%% flexibility, particularly when named citation styles are used, but
%% it also has useful features for plain and those of that ilk.
%% The natbib package gives you the following definitons
%% that extend the simple \cite:
%   \citet{key} ==>>                Jones et al. (1990)
%   \citet*{key} ==>>               Jones, Baker, and Smith (1990)
%   \citep{key} ==>>                (Jones et al., 1990)
%   \citep*{key} ==>>               (Jones, Baker, and Smith, 1990)
%   \citep[chap. 2]{key} ==>>       (Jones et al., 1990, chap. 2)
%   \citep[e.g.][]{key} ==>>        (e.g. Jones et al., 1990)
%   \citep[e.g.][p. 32]{key} ==>>   (e.g. Jones et al., p. 32)
%   \citeauthor{key} ==>>           Jones et al.
%   \citeauthor*{key} ==>>          Jones, Baker, and Smith
%   \citeyear{key} ==>>             1990

\usepackage{natbib}
%\usepackage[round,colon,authoryear]{natbib}
\setlength{\bibsep}{0pt}
\bibliographystyle{plain}
%\bibliographystyle{apalike}

\usepackage[T1]{fontenc}

% my packages
\usepackage{csquotes}

\usepackage{color}

\usepackage{tikz}
\usetikzlibrary{shapes.geometric, arrows}

\usepackage{hyperref}
\hypersetup{colorlinks, citecolor=black, filecolor=black, linkcolor=black, urlcolor=black}

\usepackage{enumitem}
\setlist{noitemsep}

\usepackage{listings}
\usepackage[T1]{fontenc}
\usepackage{lmodern}
\definecolor{gray}{rgb}{0.4,0.4,0.4}
\definecolor{darkblue}{rgb}{0.0,0.0,0.6}
\definecolor{cyan}{rgb}{0.0,0.6,0.6}
\lstset{
  basicstyle=\ttfamily\tiny,
  columns=fullflexible,
  showstringspaces=false,
}
\lstdefinelanguage{XML} {
  morestring=[b]",
  morestring=[s]{>}{<},
  morecomment=[s]{<?}{?>},
  stringstyle=\color{black},
  identifierstyle=\color{darkblue},
  keywordstyle=\color{cyan},
  morekeywords={primary,secondry,value}
  breaklines=true
}

\usepackage{multicol}

\usepackage[toc,page]{appendix}

\usepackage{graphicx}
\graphicspath{ {images/} }

% my commands
\newcommand{\nocontentsline}[3]{}
\newcommand{\tocless}[2]{\bgroup\let\addcontentsline=\nocontentsline#1{#2}\egroup}

\title{An argumentation-based approach to summarizing discussions}
\author{Charlie Egan}
% IMO this is a bit silly, but some like to include these. To remove,
% delete this declaration and remove the option from the
% \documentclass definition above.
%\qualifications{PhD, Computer Science, University College London, 1997\\%
%BEng (Hons.) Electrical and Electronic Engineering, The University of Wales, Swansea, 1992}
\school{Department of Computing Science}

%%%% In the final submission of a thesis, this should only be the year
%%%% of submission.  However, it is useful to use \date{\today} for drafts so that
%%%% they don't get mixed up.

\date{2016}

%% It is useful to split the document up as chapters and include
%% them.  LaTeX will sort out all the numbering and cross-referencing
%% for you --- if you run it enough times!

%% If you want to include only a couple of chapters then use the
%% \includeonly{} command with a list of the file/chapter names that
%% you wish to include.  NB, this must be in the preamble.

\def\sfthing#1#2{\def#1{\mbox{{\small\normalfont\sffamily #2}}}}

\sfthing{\PP}{P}
\sfthing{\FF}{F}

%% This will make sure that all cross-references are correct (including
%% references to those file not included) but will produce a dvi
%% file with only those files/chapters you specify included.

\begin{document}

%%%% Create the title page and standard declaration.

\maketitle
\makedeclaration

%%%% Then the abstract and acknowledgements

\begin{abstract}
  With an ever increasing number of internet users, online communities are hosting a ever growing number of discussions. This user generated content contains millions of statements, opinions and ideas. Current systems for discovery of such information are playing catch-up and in the meantime much of this resource is all but inaccessible.

  In this paper an approach for summarizing such information is proposed. Our approach is based on the idea of `point extraction', where a point is a verb and it's associated arguments. We implement and test our approach on six online debates before evaluating our summaries against a baseline. We found that we were able to improve significantly on a statistical extractive baseline.
\end{abstract}

\begin{acknowledgements}
  \begin{itemize}
    \item{Adam \& Advaith}
    \item{Parents}
    \item{Department}
  \end{itemize}
\end{acknowledgements}

%%%% It should have a table of contents, but delete the other two as
%%%% necessary.

\tableofcontents
%\listoftables
%\listoffigures

\chapter{Introduction\label{chap:introduction}}
  More people are online than ever before. Comment threads and forums allow us to spend our spare time participating as contributors - rather than just consumers consumers. From the latest blockbuster title to yesterday's celebrity misdemeanor there's an online conversation already well underway.

  These discussions, where statements are encoded in natural language, represent a large untapped resource of ideas and opinions. A higher level view of this ever-evolving `data set' would be of interest to many working the social sciences as well as the participants of online discussions.

  This project explores an approach for summarization of such information. At the core of the approach is the notion of a `point' - a short refined argument statement. An argument is made up of a number of points and we are going to use these as our content units in our summarization task. Points are extracted from text and grouped to give a summary of the discussion. We test our implementation's performance by running an evaluation using summaries generated from various political debates sourced from online discussions \cite{walker2012corpus}.

  \section{Motivation}
    Summarization has been a long-running task in the field of Natural Language Processing. Summarization sub-tasks such as extraction and compression, where text is selected and removed to arrive at a summary, have become commonplace due to the complexities introduced by abstractive methods. Extractive methods for this have become largely statistical using Naive-Bayes \cite{kupiec1995trainable} approaches and, more recently, Neural Networks \cite{svore2007enhancing}.

    Argumentation Mining, a newer area of study, has the aim of detecting argumentative discourse structure in text. Argumentation Mining has been successfully used in the processing of formal texts such as parliamentary records \cite{palau2009argumentation} and legal documents (cite: Semantic Processing of Legal Texts) where arguments are often stated more explicitly. However, Argumentation Mining has also more recently been applied to more informal text \cite{park2015conditional}. Such applications, coupled with summarization, encapsulates much of the idea for this project.

    The points idea was adopted from a previous project from the department that used points in a system for stance classification. This had point identification, however, the tool was not capable of extracting units of text smaller than sentences. The tool was capable of linking contrasting points, but could only make matches based on the present of negation terms.

    The project was based on the concept of a point as well as the idea that informal argumentative discourse could be used to build a high-level summary of an online discussion. News article comment sections; forum threads; film \& product reviews and even extended email conversations are all candidate applications for such a tool.

  \section{Objectives}
    Starting with the definition for a point: a verb and it's dependents, we set the following objectives for the project.

    Leading on from the stance classification project (cite Angrosh?), we wanted to improve on the extraction of points from text. This meant ensuring a complete list of a verb's dependents was maintained for presentation for a given point - rather than just using the containing sentence.

    Another goal was to investigate relationships between points such as contrastive or co-occurring points - this involved expanding the ways in which related points could be matched, beyond negation. As secondary objective, we wanted to investigate supporting points. These were points that not only commonly co-occured in posts but also had a place in an argument structure.

    The task of stance classification was also discussed. Another secondary objective was to investigate if certain points were representative for a given known stance. Stance was annotated on some of the corpus debates.

    Finally we set out to present this information in in a way that was easy to interpret. This `presentation form' evolved into our debate summaries.

\chapter{Background \& Related Work\label{chap:background-related}}
  \section{Background}
    This project is built on existing technologies and research. This section gives an overview of the key foundations.

    \subsection{Automatic Summarization}
      ``A summary can be loosely defined as a text that is produced from one or more texts, that conveys important information in the original text(s), and that is no longer than half of the original text(s) and usually significantly less than that.'' \cite{radev2002introduction}

      Approaches to summarization could be grouped loosely into two groups, extractive and abstractive. Extractive summaries are constructed using content found in the source text. Abstractive summaries generate the summary content by performing some form of analysis on the source text. One might also group summarization tasks based on the nature of the source text. Approaches tend to summarize either single or multiple documents.

      This project is more closely related to the summarization of multiple documents, however, it applies both elements of extraction and abstraction.

      \subsubsection{Multi-document Summarization}
        Summarizing text made up of documents from different authors poses an interesting challenge. Using multiple documents introduces repetition and contradictions - this makes selection tasks such as extraction and compression more complex.

        Earlier work on multi-document summarization suggested the task was going to require an abstractive approach \cite{McKeown1999TMS315149315355}. The justification being that extraction techniques used for single document summaries, not being able to connect information between documents, would lead to incoherent and repetitive summaries. An alternative approach clustered paragraphs on the same topic from multiple documents and then use natural language generation techniques to combine phrases from paragraphs into a coherent summary \cite{McKeown1999TMS315149315355}. A similar approach creates clusters by parsing sentences and using predicate-argument structures to link phrases \cite{barzilay1999information} (comment about being similar to our points extraction and grouping?). Interestingly, while both rely methods rely on natural language generation, neither use a semantic representation.

        More recently, MEAD, a centroid-based, multi-document summarization tool has been able to produce good summaries without abstraction and generation \cite{radev2000centroid}. Compared to previous abstractive systems such as SUMMONS that relied on templates \cite{mckeown1995generating}, MEAD was more generally applicable.

        Most work on the summarization of multi-document sources has focused on news articles. While there are parallels between this and the summarization of online discussion, fundamentally they are different tasks. More closely related to this project is opinion mining.

        Opinion mining or sentiment analysis is an approach often applied to user reviews, documents discussing a product or service. Early work in the area focused on the problem as a classification task, categorizing documents based on their sentiment bearing terms \cite{turney2002thumbs}. More recently there as been a greater focus on aspect-based approaches that attempt isolate sentiment terminology to specific topics such as product features \cite{hu2004mining}.

      \subsubsection{Sentence Compression \& Content Unit Size}
        While sentences are often the default `content unit' used in extractive summarization, often smaller units are more desirable. Sentences found in real documents and discussions can make a number of statements each of which can be useful when creating a summary. Sentences may also include surplus information. In this case sentences can be compressed statistically, in a similar way to documents are in extractive summarization.

        Both noisy channel and decision based models have used on parsed sentences for this task \cite{knight2000statistics}. Our approach for compression where points are extracted around a verb, does not seem to have been previously documented.


      \subsubsection{Evaluation of Summaries}

    \subsection{Argument Mining}
      Argumentation Mining is a task that involves identifying components of arguments within text such as premises and the conclusion. Part of the task often involves fitting these into a template or known pattern to enable some level of reasoning.
      \subsubsection{Identifying Argumentative Structures in Text}
      \subsubsection{Argument Relationships}
    \subsection{Dependency: Topic Modeling}
    \subsection{Dependency: Probabilistic Parsers}
    \subsection{Dependency: Graph Traversal}


  \section{Related Work}
    Overview of related work on the topic that's also related to my project
    \subsection{Opinion Mining and Summarization of Discussions}
    \subsection{Multi-document Summarization}
    \subsection{Informal Argumentation}
    \subsection{Stance Classification}
  \section{Motivation}
    Round up why our project is different and interesting

\chapter{Technologies\label{chap:technologies}}
  The project relies on a number of software dependencies, these are act as a foundation to the natural language processing services in the tool. We also use the CoreNLP framework, this just of the use of this is discussed in the Point Extraction section (ref: chapter).

  \section{Cypher and Neo4j}
    \blockquote{Neo4j is a highly scalable native graph database that leverages data relationships as first-class entities, helping enterprises build intelligent applications to meet today's evolving data challenges.} \footnote{http://neo4j.com/}

    We use Neo4j to store parse information. Sentences are parsed using the CoreNLP dependency parser and the resulting graph structure is saved into a Neo4j graph database instance. Words are represented as nodes and dependencies as edges. Nodes are are used to store the word in plaintext; the lemma, part-of-speech tag and it's index in the source sentence. Edges are directional and run from governor to dependent tokens. Edges have a string label, this is their only attribute.

    \blockquote{Cypher is a declarative graph query language that allows for expressive and efficient querying and updating of the graph store.} \footnote{http://neo4j.com/docs/stable/cypher-introduction.html}

    Cypher is used to query dependency parses currently stored in the Neo4j database. After parsing the sentences from a user's post all the dependency parses are saved. At this point a series of cypher queries are executed to filter for allowed point patterns. These patterns, derived from verbnet frame information, are represented in the Cypher syntax. Using Neo4j and Cypher allow points to be extracted from many sentences at once. This was key to completing the analysis of large debates in reasonable time.

  \begin{itemize}
    \item{differ}
    \item{erb templating}
    \item{Ruby \& Go}
    \item{docker \& services}
  \end{itemize}

\chapter{Research Goals\label{chap:res-goals}}
  Having given an overview of the related work and background to the project, we now go on to outline the project's goals.

  Building on points extraction, we wanted to use extracted points as a new type of content unit in summarizing large discussions. Points allow sentences to be broken down into a number of atomic units that can be compared and matched to other points. We wanted to use points, and the analysis they enable, to build a picture of the discussion.

  This first required a robust means of identifying and extracting points from text to serve as a foundation for later analysis. We needed to investigate not only whether points were a viable content unit, but if they could succinctly present information better than extracted sentences. Given plain text from a discussion, we would need 1) a pattern or signature that could be used to link points --- regardless of their exact phrasing --- and 2) a shorter extract that could be used to present the point to readers.

  We were also interested to build on these extracted points by linking them in different ways to model the discussion. Firstly, matching points and counter points using negation and antonyms was the main comparison of interest. Additionally, linking co-occurring points as a more generalized view on participant stances was also a goal. We wanted to use these relations between points; as well as point metadata extracted from text such as referenced topics, source post and negation; in formulating a structured summary that is useful to readers.

  We also needed to evaluate the results of the tool as a whole. Testing the value of our analysis, and whether the results were useful, was also a goal for the project. We wanted to show that our approach to points extraction was suitable for the summarization of discussions. Aware that the base analysis of extracting points has utility beyond that of summarizing discussion, we wanted to build our tools for analysis in a modular manner such that components may be reused in the future for further work on other tasks.

\tikzstyle{service} = [rectangle, minimum width=2cm, minimum height=1cm, text centered, draw=black, fill=white]
\tikzstyle{extservice} = [rectangle, minimum width=2cm, text width=1.9cm, minimum height=1cm, text centered, draw=black, fill=gray!50]
\tikzstyle{flow} = [thick,->,>=stealth]
\tikzstyle{depends} = [thick,-,>=stealth]

\chapter{System Architecture\label{chap:system-architecture}}
  This chapter is intended to give a technical overview of the system and give context to the analyses detailed in the following chapters. The system has been implemented as a series of isolated services that operate in a processing pipeline. The architecture is the product of the experimental development process, features were implemented before being wrapped in a interface for reuse by future components. Data is transformed at three stages in what could be described as an analysis pipeline. Each stage takes significant time to complete so the output is written to disk in shared format between stages.

  \begin{figure}[!h]
    \centering
    \begin{tikzpicture}[node distance=3.5cm]
      \node (agg) [service] {Aggregator};
      \node (top) [service, left of=agg] {Topic Analyzer};
      \node (ext) [service, above of=agg, yshift=-1.5cm] {Extractor};
      \node (cur) [service, right of=agg] {Curator};
      \node (sum) [service, right of=cur] {Summarizer};
      \node (cor) [extservice, above of=sum, yshift=-1.5cm] {CoreNLP};
      \node (neo) [extservice, above of=ext, yshift=-1.5cm] {Neo4j};

      \draw [flow] (agg) -- (cur);
      \draw [flow] (cur) -- (sum);
      \draw[dotted] [depends] (agg) -- (ext);
      \draw[dotted] [depends] (ext) -- (neo);
      \draw[dotted] [depends] (ext) -- (cor);
      \draw[dotted] [depends] (sum) -- (cor);
      \draw[dotted] [depends] (top) -- (agg);
    \end{tikzpicture}
    \caption{\\Architecture Diagram. Arrows represent the pipeline; dotted lines represent dependencies.}
    \label{fig:arch-dia}
  \end{figure}

  Figure \ref{fig:arch-dia} shows a high-level overview of the various modules and how these build up the analysis pipeline. Points for the source text are gathered by the \textit{Aggregator}, these are filtered and cleaned by the \textit{Curator}. Finally curated points are used to generate a summary by the \textit{Summarizer}.

  Before source text can be analyzed it must be cleaned for invalid characters. Posts in a discussion start as single text files, these are loaded and transformed into JSON files that preserve their index and valid UTF-8 content. The Aggregator loads these JSON post files to extract the points made in the discussion.

  The Aggregator loads all the JSON posts for the selected discussion. Before extracting points for each in turn it uses the text from the posts to get a list of topic words from the \textit{Topic Analyzer}. Topic words are included when making requests for post points, only sentences containing topic words are analyzed. For each post in turn the text and topics are submitted to the \textit{Extractor}. This service carries out the point extraction process using both the CoreNLP dependency parser and Neo4j datastore - this is described in detail in the following chapter. All extracted points for all posts are saved to disk. A modern, mid-range computer can complete 150,000 words spread across 2300 posts in around 10 minutes, this is the slowest stage in the pipeline.

  While points must come from a sentence containing a topic word, at this stage points are still largely unfiltered. The \textit{Curator}, a short program to filter and re-format points is the next stage. This stage makes a series of transformations on points such as merging pronouns under a single \texttt{PERSON} label. Patterns for low-quality points are also encoded at this stage, for example: the \texttt{PERSON.nsubj be.verb sure.adj} point that comes from every ``\textit{I'm sure}''. These transformations and rejections are applied to the list of points and a refined version is saved for use at the next stage.

  The \textit{Summarizer} is the final stage in the pipeline. Given a list of points, this module groups common points into different the different summary sections. Sections are filled using available, qualifying points. Once a point has been spent on section, it cannot be used again. This means that should a point like \texttt{life.nsubj begin.verb at.prep conception.dobj} be used in the section for contrasting points it cannot be used in a later section. Even should \textit{life} be a used in the later ``commonly discussed topics'' section, \texttt{life.nsubj begin.verb at.prep conception.dobj} will not be used.

  This module must also choose an extract to display for each point. The point \texttt{abortion.nsubj be.verb right.dobj} occurs many times in the discussion and a single occurrence to represent this must be selected. This process makes use of the CoreNLP service again to request a dependency parse used in rating the quality of extracts. Selected extracts are formatted and used as text in summaries.

  The \texttt{Summarizer} formats the grouped extracts for selected points as summary HTML files - a range of formats are currently exported for use in the evaluation questionnaires. This is currently the end of the pipeline. Possibilities for further work are discussed in the chapter \ref{chap:conclusion}.

  A monolithic architecture was avoided due to processing time required at each stage. Running the entire process when making adjustments at the final stage was not feasible. Currently stages in the pipeline must be run manually however were the pipeline to be used in it's entirety this could be automated. The process for using the pipeline has been outlined in the maintenance manual \textbf{ref}.

  This is intended to give an overview of the system and how the data is transformed at each stage. The following chapters discuss each module in detail explaining the intricacies of our analysis approach.

\chapter{Point Extraction\label{chap:point-extraction}}

\chapter{Point Curation\label{chap:point-curation}}
Points extracted at the first stage must be refined to make them suitable for use in the following summarization module. These refinements focus on removing points generated from common phrases such as \textit{`I'm sure'} and \textit{`Ask yourself\dots'}. While adjustments could be made as points are extracted, this was implemented as a separate task during our implementation phase. Rejecting and editing point content is also conceptually different from identifying them in source text. Much additional information is encoded at this stage.

  \section{Minimum Requirements}
    Some basic restrictions are applied to points matched by the Generic Frame. The Generic Frame returns a large number of additional points (on top of other frame results) many of which are of little interest. To make sure that points from the Generic Frame are useful the following heuristics are applied as requirements.

    \begin{itemize}
      \item{Contain a topic word - this is a good estimation of a point's informativeness}
      \item{Have a subject, object and average word length of 5 or more characters - this restricts extremely short points from being matched. Points ideally make some form of connection between words of interest}
    \end{itemize}

    These requirements balance the benefits and problems of matching a greater number of patterns, some lower quality, using the Generic Frame. This means that shorter points, or points without a search topic word can only be returned by matching the query for a FrameNet frame.

  \section{Person Nominal Subjects}
    In an effort to better cluster extracted points into distinct statements used in the discussion, we opted to merge pronouns under a single Person subject. Points with person nominal subjects are very common and often segregated by different pronouns. Take the following example:

    \begin{itemize}
      \item{\texttt{people.nsubj kill.verb people.dobj}}
      \item{\texttt{he.nsubj kill.verb people.dobj}}
      \item{\texttt{they.nsubj kill.verb people.dobj}}
    \end{itemize}

    While there is a semantic difference between these points, we made the decision to amalgamate them under patterns like \texttt{PERSON.nsubj kill.verb people.dobj}. This means clusters for points are not dispersed among many pronouns. Reducing repetition between clusters means we can continue to rely on a cluster's size as a measure of salience in the summarization task.

  \section{Blacklists}
    To complete this module's function of cleaning and refining points, a number of `blacklists' were established. These defined generic point patterns that were of little interest and should be removed from all topics of discussion. Blacklists were based on the patterns of points extracted from the Abortion debate corpus. All blacklisted patterns are listed in Appendix \ref{app:blacklists}.

    \tocless\subsection{Nominal Subjects}
    First points with pronoun or determiner nominal subjects were rejected. Points such as \texttt{it.nsubj have.verb rights.dobj} are ambiguous as the surrounding sentences are required to resolve the anaphor \blockquote{it}. One possible resolution was to merge such points with the most common pattern with a noun or proper noun nominal subject (\texttt{fetus.nsubj have.verb rights.dobj} for example). Rather than merging ambiguous points into other groups, we opted to remove them in the interest of keeping the results verifiable.

      Currently the following subjects are grounds for rejection: \textit{it(PRP), that(DT), this(DT), which(WDT), what(WDT)}. Other words of the same tag such as \textit{they} or \textit{his/her} but which did not occur as a prominent problem group are not excluded. Rather than prematurely removing content, these are currently permitted.

    \tocless\subsection{Person Verb Combinations}
    There are a series of verbs that are not permitted to form points when they have a \texttt{PERSON} subject. Certain phrases such as \blockquote{\textit{I think}} as very common; while important for comprehension, they do not make interesting points. \textit{object, understand, realize, debate, speak, stand, refer, explain, support, feel} are some examples of `actions' that participants write about which generate unsuitable points for use in summaries.

      These restrictions are only applied to points that consist of two components, a verb and \texttt{PERSON} subject. This means that \texttt{PERSON.nsubj understand.verb issue.dobj} is a valid point, whereas \texttt{PERSON.nsubj understand.verb} is not.

    \tocless\subsection{Patterns}
      Common phrases create another class of poor quality points. Phrases such as \textit{`I'm correct [in saying]', `Something happened [to/that]', `[opposition] make the claim'} do not add valuable information. They could not be used to add useful content to the summary and are removed.

\chapter{Summary Generation\label{chap:summary-generation}}
  A process to extract and refine a list of points could be used in a number of ways. From this foundation, we investigate a potential use for points in a summarization task. Extracted points are short and have a `pattern' that allows for new comparisons not possible with plain text extracts. This chapter first explains a number of shared components used in building each summary section, then goes on to describe each section in detail.

  \section{Permitted Extracts}
    In a cluster of points with the same patter there are a range of different extracts that could be selected. There is much variation in quality caused by additional words, bad parses and poor punctuation. Removing points with extracts that matched heuristics for poor quality would significantly reduce the information available to determine a cluster's salience. Such points are useful in aggregation but not for presentation.

    Instead we implemented a set of rules that prevent a point's extract from being used in the summary, even while the point itself remains in the cluster to aid aggregation and sorting tasks.

    Predominantly exclusions are made based on the presence of certain substrings using regular expressions. Exclusion patterns include: starts with a wh-question word or contains a question mark; has two or more consecutive words in block capitals or contains a mid-word case change. Since these patterns can be applied very quickly extracts are tested using these first.

    Following on from these, there are other more complex exclusion patterns based on the dependencies from re-parsing the extract using CoreNLP. Extracts with clausal or generic dependencies are excluded, such dependencies are characteristic of long points or erroneous parses. An extract may also be excluded if it contains more than two conjunctive relations or nominal dependencies.

    Finally some additional restrictions, that are not based on the dependencies or the strings, are applied. For example, an `it' must be preceded by \{on, and, but, whether\} and words are not allowed to be repeated.

  \section{Extract Selection}
    Points are manipulated in clusters where members share a common pattern, for example, there might be a cluster where each point had the pattern \texttt{woman.nsubj make.verb choice.dobj}. This cluster would contain all the points with this pattern and in turn all the extracts that could be used in a summary to express this point. Even after refining the list of points input to this component of the pipeline there is much variation in the quality of the extracts available for selection. Take this example cluster of extracts for points about the Genesis creation narrative:

    \begin{itemize}[label={}]
      \item{\blockquote{The world was created in six days.}}
      \item{\blockquote{The world was created in exactly 6 days.}}
      \item{\blockquote{Is there that the world could have been created in six days.}}
      \item{\blockquote{The world was created by God in seven days.}}
      \item{\blockquote{The world was created in 6 days.}}
      \item{\blockquote{But, was the world created in six days.}}
      \item{\blockquote{How the world was created in six days.}}
    \end{itemize}

    All of these passed the `Permitted Extracts' stage. The task is to select the best extract to represent the cluster. In this instance our approach selected the fifth point, \blockquote{The world was created in 6 days.}. This selection is made every time a cluster has been selected for use in a summary as an extract to represent it is required. This is done using a bag-of-words, length-weighted, bigram model of the extract words in the cluster. The goal was to select the most succinct extract that was most representative of the pattern cluster.

    Extracts are iterated and the bigrams for each collected. Bigrams are given a value equal to the number of times they occur in the cluster. Each extract is then given a value equal to the sum of the values for the bigrams it contains. This extract value is then divided by the number of words in the extract to determine a final score. The extract with the highest score is selected for use in the summary.

  \section{Extract Presentation \& Formatting}
    Our points extraction approach works by selecting the relevant components in a string for a given point, using the dependency parse graph. While this has a key advantage in creating shorter content units, it also means that extracts are often poorly formatted for presentation in isolation. To overcome this we needed to implement a means of correcting extracts for presentation, below are two examples that illustrate the problem with poor punctuation and style.

    \blockquote{that Scientifically , human life begins at this stage}
    and
    \blockquote{.that a fetus is a person ,}

    To overcome this we have a short function to ensure the extract meets the following properties: extract is capitalized; ends in a period; commas are not preceded by a space; contractions are applied where possible and consecutive punctuation marks condensed or removed. Certain determiners, adverbs and conjunctions (because, that, therefore) are also removed removed from \textit{the start} of extracts. The two extracts above are translated into the following cleaner versions:

    \blockquote{Scientifically, human life begins at this stage.}
    and
    \blockquote{A fetus is a person.}

    This particular feature still requires work. The task of correctly formatting a string has only been touched on here and is a candidate for further work on the project.

  \section{Avoiding Repetition in Summaries}
    A cluster's inclusion in a particular summary section is a function of the number of points in the cluster. This is based on the idea that larger clusters are have greater importance and should the highest priority to include in the summary. To avoid large clusters being repeatedly selected at each summary section, a list of used patterns and extracts is maintained.

    When an extract is used in a summary section it is `spent', and added to a list of used extracts. The extract pattern, string, lemmas and subject-verb-object triple are added to this list. Any point that matches anything in this list of used identifiers cannot be used. The same extract may be included in many clusters under different point patterns. For example, \blockquote{Life begins at conception.} is matched by both the \texttt{life.nsubj begins.verb} and \texttt{life.nsubj begins.verb at.prep conception.dobj} patterns. This means that when checking is an extract is suitable it must be checked against the text \textit{and} patterns of those already used.

    \tocless\subsection{Negation Shorthand}
      As part of the counter point analysis discussed in the following section, a condensed point representation was developed for expressing points that shared many of the same words. For example, to include both \blockquote{A fetus is a human} as well as \blockquote{A fetus is \textbf{not} a human} is a highly repetitive presentation. Such a presentation does not make for good reading and uses words that could be used to express additional information from another point.

      Based largely on a string diffing library\footnote{https://rubygems.org/gems/differ}, we developed an alternative, more condensed presentation for such pairs of points. Under this, the above examples could be written as \blockquote{A fetus is \textbf{\{}not\textbf{\}} a human}, saving five repeated words. However, more complex examples also occurred such as \blockquote{Abortion \textbf{\{}is not\textbf{|}should be\textbf{\}} legal \textbf{\{}after 12 weeks\textbf{\}}} (\blockquote{Abortion should be legal} and \blockquote{abortion is not legal after 12 weeks}). While there are adjustments that can be made to improve this, such as requiring the two extracts to have a similar number of words, complex examples still surfaced and the formatting was not used.

      The diffing functionality is still used in the identification of counter points.

  \section{Summary Sections}
    \subsection{Counter Points}
      The identification of counter points were a goal for the project from the start. Counter points are matched on one of two possible criteria, either the presence of negation terminology or a antonym for a work that is part of the point's pattern.

      Potential, antonym-derived counter points, for a given point, are generated using it's pattern. Taking the pattern \texttt{treatment.nsubj be.verb private.dobj} as an example, this generates the counter point \texttt{treatment.nsubj be.verb \textbf{public}.dobj}. These substitutions can be applied to any component that has an antonym, including verbs (\texttt{sperm.nsubj have.verbs rights.dobj} vs \texttt{sperm.nsubj \textbf{lack.verb} rights.dobj}). If there are many words in the pattern with antonym matches then multiple potential counter points are generated for a single point.

      Once a list of potential point vs. counter point patterns has been generated these are filtered to make sure that both the generated counter point pattern exists in the list of points and that the there are no duplicate pairs. A point/counter point pair cannot also appear in in reverse as part of another pair.

      Counter points are selected for display based on the average size of the clusters for the point and counter point. This is intended to represent ``the most common pairs of contradictory points''. In the summary text only the extract for the point is displayed, not the counter. The section is introduced as ``points that people disagree on'' so stating either point presents it as a point that is contested. Previously the `negation shorthand' was used to present points and counter points but variation in extracts often led to complex examples.

    \subsection{Negated Points}
      Negated points are displayed in the same summary section, `points that people disagree on'', however they are identified differently. Negated points are selected from the largest unused clusters. Negation terminology is not commonly part of the cluster pattern, for example, the \texttt{woman.nsubj have.verb right.dobj} cluster could include both \blockquote{A woman has the right} and \blockquote{The woman does not have the right} as extracts. Identifying these negated forms within clusters is how negation-derived counter points are collected.

      First the cluster is split into two groups, extracts with negation terminology and those without. The Cartesian product of these two groups gives all pairs of negated and non-negated extracts. For each of these pairs; a string difference is computed\footnote{https://rubygems.org/gems/differ}, the complexity of this difference pattern is used to identify a clean match. For example \blockquote{\{+that \}a fetus is \{-not \}a person\{- under the law of the eu\}} is not as clean as \blockquote{that a fetus is\{n't \}a person}. By counting elements in the difference pattern we are able to select shorter and cleaner examples where they exist.

      However, similarly to antonym-derived counter points, clean examples do not always exist. Negated points are represented as the positive side extract of the pair and listed under the ``points that people disagree on'' section.

    \subsection{Co-occurring Points}
    \subsection{Commonly Occurring Points}
    \subsection{Longer Pattern Points}
    \subsection{Topic Points}
    \subsection{Topic Linking Points}
    \subsection{Questions}

\chapter{Evaluation\label{chap:evaluation}}
  This chapter will detail the evaluation metrics, how they were tested and what the results of the evaluation were. Two (or three) sets of evaluations were carried out making use of six debates from our corpus \cite{walker2012corpus}: abortion, creation, gay rights, the existence of god, gun ownership and healthcare.

  \section{Research Questions}
    We wanted to evaluate our tool as a tool for automated summarization. The result of a final analysis was a summary comprising of point extracts for a given debate, by assessing the quality of this we also evaluate the precursor analysis. The following research questions were set:

    \begin{itemize}
      \item{Are our summaries more readable and informative than those produced by existing tools?}
      \item{Does introducing summary sections with an explanation improve readability?}
      \item{What is the level of agreement between our extract selection module and humans given the same task?}
      \item{TBD: Which sections of generated summaries are most useful to readers?}
    \end{itemize}

    To address these questions we compared different versions of our summaries against equal-length summaries generated by an implementation of \cite{nenkova2006compositional} (`stock summaries'). We also gathered scores for groups of extracts and compared summaries generated with randomly selected extracts against one based on our bigram model for selection.

  \section{Design}
    In response to the our research questions we set the following null hypotheses for the evaluation:

    \begin{itemize}
      \item{The readability and informativeness of our summaries do not improve on those of existing tools. (\textbf{H1})}
      \item{Explanations introducing summary sections does not improve readability. (\textbf{H2})}
      \item{Our means of selecting extracts is not better than random selection. (\textbf{H3})}
      \item{\textcolor{red}{TBD: All summary sections are equally useful to readers. (\textbf{H4})}}
    \end{itemize}

    To test these hypotheses we prepared three comparative studies. These are described in the following sections.

    \tocless\subsection{Summary Styles}
      In the upcoming sections, reference is made to a range of summary styles.
      \begin{itemize}
        \item{\textbf{Stock}: A summary generated using an implementation of \cite{nenkova2006compositional}, used as a benchmark for summaries generated by our tool. There is no structure to these summaries.}
        \item{\textbf{Plain}: This is a collection of point extracts in the same styles as the stock summaries. The order of extracts is the same as it would be if the summary had sections annotated. This presentation is designed to be as close as possible to the stock summary presentation.}
        \item{\textbf{Layout}: A summary that has explanatory text that introduces sections. The extracts are the same as those in the plain summary.}
        \item{\textbf{Formatted}: A layout summary with explanation keywords in bold and topic words in green.}
      \end{itemize}

      Participants for Study 1 and 2 were recruited using Amazon Mechanical Turk. \textcolor{red}{In the third study\dots}

    \tocless\subsection{Study 1: Summary Comparison}
      In order to address \textbf{H1} and \textbf{H2} we designed a questionnaire that compared various summary styles against equal-length stock summaries. Each questionnaire was made up of three sections.

      The first section compared a plain summary against a stock summary, on the same topic. Users were instructed to read both summaries and rate them relatively on the following criteria: `Content Interest / Informativeness', `Readability', `Punctuation \& Presentation' and `Organization'. Finally they they were asked to give an overall rating and justify their response. The next section asked the same questions but instead compared a layout summary against a stock one. Layout summaries are longer because of the explanatory text, the stock summary was the same length as the layout summary and was longer than the stock summary used in the first section. The final section compared a layout summary with a formatted one. The layout summary is reused from the second section, the formatted summary is also the same content (only different formatting). There was only an overall rating and justification for the comparison in this section.

      There were six versions of the questionnaire setup in a Latin square to cover all six topics. Section 1 had a different topic from sections 2 so as to not repeat summary content. Section 3 addresses the question of formatting only and has the same content as the layout summary in section 2 to make the task faster.

      Responses from the first section that compare our plain summary against a stock one can be used in addressing \textbf{H1}. The difference in the responses gathered in section 2 will allow us to address \textbf{H2}. Section 3 extends on this, testing if the formatting is a useful addition to the layout summary.

    \tocless\subsection{Study 2: Extract Comparison}
      To investigate the performance of the tool in greater depth we ran a further study testing the quality of our extract selection mechanism. Responses from this questionnaire addressed \textbf{H3}.

      The task given to participants had two sections. First were a series of extracts for a number of point patterns, participants were asked to rate extracts accounting for their succinctness and the extent to which they made sense. This was the closest we could make the task for participants to that of the tool. The following section, similar to the first study compared two summaries. Both summaries were of the formatted style however their content differed. The extracts in one summary were selected using the bigram model, the extracts in the other were selected at random. Participants were asked to give these a relative rating and justify their response. Responses for these two tasks were both intended to address \textbf{H3}.

    \tocless\subsection{\textcolor{red}{TBD: Study 3: Section Comparison}}

  \section{Results}
    \subsection{Study 1: Summary Comparison}
      The results of the summary comparison showed a strong preference for summaries based on points based extraction. When comparing the plain and stock summaries, our plain summaries were marked as better or much better by 69\% of participants. Plain summaries, lacking any kind of layout, was still scored better or much better by 57\% of participants.

      When comparing layout summaries against equal length stock summaries participant preference only became clearer. 89\% of participants rated layout as better or much better than stock. Organization saw the greatest change, with a 30\% increase in favor of layout summaries. The average increase across all factors was 19\%. The ratio of `better' to `much better' also shifted, `much better' ratings increased by 15\% while `better' ones decreased by 5\%.

      There was also variation between summaries of different discussions. The gay rights layout summary saw an increase of 45\% in better/much better scores while the guns layout summary saw no improvement. When comparing plain and stock summaries on guns and healthcare all participants rated plain as being the same or better. When comparing the healthcare layout and stock summaries all participants marked the layout summary as being much more readable. The plain vs stock creation summary was the most divided, 42\% of responses were marked as same or stock better.

      Despite the points used in the layout and plain summaries being the same, the `Content' factor saw an increase of 14\% in better/much better scores. Punctuation was also saw a 13\% increase, despite also being the same.

      Regarding the results of the layout / formatted summary comparison, more than half of respondents preferred the formatted version of the summary. This is significantly lower than the improvements of layout vs stock (89\%) or even plain vs stock (69\%).

      These findings are based on the 55 responses from 6 different questionnaires, the target population being people interested in online discussions. To conclude, the results show a preference for points based summaries. This can further be improved by introducing sections, and by a less decisive margin, highlighting keywords.

    \subsection{Study 2: Extract Comparison}
    \subsection{\textcolor{red}{TBD: Study 3: Section Comparison}}

  \section{Discussion}
    Based on on comments left as justification, some found the colors distracting and unnecessary as well as making the summary noisy and hard to read.
    Also need some samples of the comments that highlight interesting things

\chapter{Summary \& Conclusion \label{chap:conclusion}}
  \section{Summary}
    In this project we have implemented a robust method for extracting points, a meaningfully shorter content unit than a sentence. We make use of these in a summarization task by clustering points into cohesive sections. We then evaluate the effectiveness of our approach by comparing our summaries against summaries generated by a baseline statistical tool that used sentence extraction.

    We were able to meet most of the project's initial goals. Using the dependency parse, we have implemented a means of extracting more useful and complete point extracts than was implemented in the previous project. We also improved on the detection of counter points by expanding the approach to account for antonyms --- rather than negation alone. While we also implemented an approach for the identification of co-occurring points, the results seem to be more variable and are highly dependent on large discussions where participants make many points.

    We opted to present the results of the analysis as summaries of the debate. Here we were able to go beyond our goals of counter/co-occurring points and include additional sections based on the points we had extracted.

    There are however areas that require further work. Results from our evaluation suggest that certain presentation decisions we not universally popular. We also found that our means of selecting extracts from clusters was not significantly better than random, this appeared to be largely because points with poor readability scored well on the bigram model. \textcolor{red}{We were able to reject 1 (or 2/3) of our 4 hypotheses}. The evaluation did also present positive results for plain and layout summaries, giving justification for further work using points extraction as a basis.

    Comments from study 3, at the social media workshop, suggested that the summaries fell between quantitative and qualitative analysis and this they could be made more useful by quoting the (already available) ratio of counter-point sides. Comments also suggested it would be useful to select the required summary length as well as provide links back to the input text. The disagreement section was referenced as being the most useful while participants found related points confusing.

  \section{Further Work}
    The implementation is still the product of an experimental development process. The components of the summary generation approach have poor maintainability. Next steps include reducing the agglomerative complexity and repetition as well as establishing a basic level of unit testing. Clearer definitions for the re-formatting of extracts and extract selection would also be beneficial for code readability. This ties into the corpus pre-processing and extract presentation, both areas that require a more integrated implementation.

    The tool chain is currently closely coupled with the corpus. We would like to establish a more general approach to make analysis of new corpora less bespoke. One interesting direction would be a simple web application that capable of generating a summary for a discussion of the user's choosing from \textit{Reddit}, \textit{Hacker News} or online forum. This potential use case raises issue of the required discussion length. Currently a long discussion (upwards of 100,000 words) is required to extract a sufficient number of points to fill all our summary sections without repetition --- and build a good sample of counter/co-occurring points. This would likely require the summary structure to be more flexible, generating shorter summaries for shorter discussions and contextually removing sections where information was lacking.

    From the evaluation results we found that our bigram extract selection approach was not providing reliably better results than random. This appeared to come down to readability informativeness. Currently extracts are weighted by the length, perhaps this is dominating bigram score representing informativeness. It would be interesting to introduce readability as an additional factor, perhaps by incorporating an automated readability index. Should the tool be made accessible to the public as a web application there would also be the opportunity to use user rating of extracts to train a model for extract preference. The task of selecting readable, informative extracts that represent the cluster is an interesting task with many options for further work.

    Currently the approach is based a flat list of posts in a discussion without reply/response annotations. Using hierarchical discussion threads open up interesting opportunities for Argument Mining using points extraction as a basis. A new summary section that listed points commonly made in response to other points in other posts would be an interesting addition. This would also be possible with discussions on \textit{Twitter} where reply information is also exposed. Related to this, adding user identity information would also make it possible to track a posts by the same user in discussions, this represents another interesting area for further features.

    Another direction that would be inserting to explore would be to build a graph-like representation of the discussion. Using point's subject and object information a graph of nodes representing nouns connected by verbs as edges could be generated. This would be made more interesting if the semantic annotations from the verb frames could be included. As part of this project we experimented with a presentation of this type, an example is included in Appendix \ref{app:disc_graph}. Related to this is the idea of using the point's semantic annotations to to investigate abstractive summarization.

    There is also potential for further work on the summary interface. References back to source text could help explain points that require more context. A query interface where summaries could be tailored to set of reader-defined topics would also be interesting. This and many other possibilities for features and evaluation come from exposing and promoting this analysis as a public web application.

  \section{Conclusion}
    In this project we have shown that our points extraction is a viable foundation for summarization of online discussion. We think this success can be generalized to tasks beyond summarization that make use of text extracts. We have shown that existing tools struggle with the summarization of discussion and that summaries with sections that are designed to give a more complete overview are preferred. We see summarization as just one use case for such analysis and that there are many other varied layers that could be built on top of our points extraction implementation.

    We see this project a step forward in the process of better understanding online discussion. Our hope is that this work can become the basis for further work and applications that allow the exploration of the wealth of ideas and arguments currently hidden in archived threads.


\appendix
\include{proof}

\bibliography{general,introduction,background,related}
\begin{appendices}
  \addtocontents{toc}{\protect\setcounter{tocdepth}{1}}
    \makeatletter
    \addtocontents{toc}{%
      \begingroup
      \let\protect\l@chapter\protect\l@section
      \let\protect\l@section\protect\l@subsection
    }
    \makeatother

    \chapter{User Manual}
      \section{First section}
      \section{Second section}

    \chapter{Maintenance Manual\label{app:maintain}}
      \section{First section}
      \section{Second section}

    \chapter{Discussion Graph Representation\label{app:disc_graph}}
      \begin{figure}[h]
        \caption{A sample graph representation of the abortion discussion}
        \centering
        \includegraphics[width=0.8\textwidth]{disc_graph}
      \end{figure}

    \chapter{Blacklists\label{app:blacklists}}
      \section{Ambiguous subjects}
        \textit{it, that, this, which, what}

      \section{Disallowed Person Actions}
        The following verbs are not allowed in a 2 component point with a \texttt{PERSON.nsubj}:

        \textit{agree, argue, ask, begin, believe, believe, call, care, change, close, come, come, continue, debate, disagree, end, explain, fail, feel, feel, find, follow, get, go, go, guess, happen, hear, leave, live, lose, make, move, object, open, read, realize, refer, show, sit, speak, stand, start, support, take, talk, tell, think, try, understand, wonder, write}

      \section{Disallowed Points}
        \texttt{PERSON.nsubj be.verb} cannot be completed by: \textit{able, aware, correct, false, favor, glad, good, here, interested, likely, one, right, say, sorry, sure, true, willing, wrong}

        \noindent\texttt{PERSON.nsubj want.verb} cannot be completed by: \textit{have, what, what do}

        \noindent\texttt{PERSON.nsubj} cannot be completed by: \textit{say.verb what.dobj, mean.verb what.dobj, know.verb what.dobj, believe.verb what.dobj, see.verb what.dobj, see.verb argument.dobj, have.verb problem.dobj, tell.verb they.dobj, think.verb what.dobj, argue.verb in.prep fact.dobj, argue.verb with.prep you.dobj}

        \begin{itemize}
		  \item{debate.nsubj be.verb about.dobj}
		  \item{question.nsubj be.verb}
		  \item{make.verb claim.dobj}
		  \item{ask.verb yourself.dobj}
		  \item{thing.nsubj happen.verb}
		  \item{something.nsubj happen.verb}
        \end{itemize}

    \chapter{Summary Comparison Survey Section\label{app:survey-section}}
      \begin{figure}[h]
        \caption{An example study 1 survey section comparing a plain and stock survey.}
        \centering
        \includegraphics[width=0.9\textwidth]{survey}
      \end{figure}

    \chapter{Extract Survey Section\label{app:extract-survey-section}}
      \begin{figure}[h]
        \caption{An example study 2 survey section where participants rated extracts.}
        \centering
        \includegraphics[width=0.9\textwidth]{survey2}
      \end{figure}

  \addtocontents{toc}{\endgroup}
\end{appendices}


\end{document}
