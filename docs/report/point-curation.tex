\chapter{Point Curation\label{chap:point-curation}}
Points extracted at the first stage must be refined to make them suitable for use in the following summarization module. These refinements focus on removing points generated from common phrases such as \textit{`I'm sure'} and \textit{`Ask yourself\dots'}. While adjustments could be made as points are extracted, this was implemented as a separate task during our implementation phase. Rejecting and editing point content is also conceptually different from identifying them in source text. Much additional information is encoded at this stage.

  \section{Minimum Requirements}
    Some basic restrictions are applied to points matched by the Generic Frame. The Generic Frame returns a large number of additional points (on top of other frame results) many of which are of little interest. To make sure that points from the Generic Frame are useful the following heuristics are applied as requirements.

    \begin{itemize}
      \item{Contain a topic word - this is a good estimation of a point's informativeness}
      \item{Have a subject, object and average word length of 5 or more characters - this restricts extremely short points from being matched. Points ideally make some form of connection between words of interest}
    \end{itemize}

    These requirements balance the benefits and problems of matching a greater number of patterns, some lower quality, using the Generic Frame. This means that shorter points, or points without a search topic word can only be returned by matching the query for a FrameNet frame.

  \section{Person Nominal Subjects}
    In an effort to better cluster extracted points into distinct statements used in the discussion, we opted to merge pronouns under a single Person subject. Points with person nominal subjects are very common and often segregated by different pronouns. Take the following example:

    \begin{itemize}
      \item{\texttt{people.nsubj kill.verb people.dobj}}
      \item{\texttt{he.nsubj kill.verb people.dobj}}
      \item{\texttt{they.nsubj kill.verb people.dobj}}
    \end{itemize}

    While there is a semantic difference between these points, we made the decision to amalgamate them under patterns like \texttt{PERSON.nsubj kill.verb people.dobj}. This means clusters for points are not dispersed among many pronouns. Reducing repetition between clusters means we can continue to rely on a cluster's size as a measure of salience in the summarization task.

  \section{Blacklists}
    To complete this module's function of cleaning and refining points, a number of `blacklists' were established. These defined generic point patterns that were of little interest and should be removed from all topics of discussion. Blacklists were based on the patterns of points extracted from the Abortion debate corpus. All blacklisted patterns are listed in Appendix \ref{app:blacklists}.

    \tocless\subsection{Nominal Subjects}
    First points with pronoun or determiner nominal subjects were rejected. Points such as \texttt{it.nsubj have.verb rights.dobj} are ambiguous as the surrounding sentences are required to resolve the anaphor \blockquote{it}. One possible resolution was to merge such points with the most common pattern with a noun or proper noun nominal subject (\texttt{fetus.nsubj have.verb rights.dobj} for example). Rather than merging ambiguous points into other groups, we opted to remove them in the interest of keeping the results verifiable.

      Currently the following subjects are grounds for rejection: \textit{it(PRP), that(DT), this(DT), which(WDT), what(WDT)}. Other words of the same tag such as \textit{they} or \textit{his/her} but which did not occur as a prominent problem group are not excluded. Rather than prematurely removing content, these are currently permitted.

    \tocless\subsection{Person Verb Combinations}
    There are a series of verbs that are not permitted to form points when they have a \texttt{PERSON} subject. Certain phrases such as \blockquote{\textit{I think}} as very common; while important for comprehension, they do not make interesting points. \textit{object, understand, realize, debate, speak, stand, refer, explain, support, feel} are some examples of `actions' that participants write about which generate unsuitable points for use in summaries.

      These restrictions are only applied to points that consist of two components, a verb and \texttt{PERSON} subject. This means that \texttt{PERSON.nsubj understand.verb issue.dobj} is a valid point, whereas \texttt{PERSON.nsubj understand.verb} is not.

    \tocless\subsection{Patterns}
      Common phrases create another class of poor quality points. Phrases such as \textit{`I'm correct [in saying]', `Something happened [to/that]', `[opposition] make the claim'} do not add valuable information. They could not be used to add useful content to the summary and are removed.
