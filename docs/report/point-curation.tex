\chapter{Point Curation\label{chap:point-curation}}
  For the points extracted at the first stage in the pipeline to be useful in the summarization module there are a number of selections and refinements that must be made. These refinements focus on removing points generated from common phrases such as `I'm sure' and `Ask yourself\dots'.

  \section{Blacklists}
    To complete this selection, a number of `blacklists' were established. These defined patterns and components that should be removed before proceeding.
    \subsection{Nominal Subjects}
      First points with pronoun or determiner nominal subjects were rejected. Points such as \texttt{it.nsubj have.verb rights.dobj} are ambiguous as the surrounding sentences may be required to resolve the anaphor \textit{it}. One possible resolution was to merge such points with the most common pattern with a noun or proper noun nominal subject. However, since groups of points were already of additional size we opted to remove these points instead to help keep results verifiable.

      Currently the following subjects are grounds for rejection: \textit{it(PRP), that(DT), this(DT), which(WDT), what(WDT)}. Other words of the same tag such as \textit{they} or \textit{his/her} have not yet been excluded as they have yet to be the basis for an ambiguous point.

    \subsection{PERSON Verb Combinations}
      There are a series of verbs that are not allowed to form points when they have a \texttt{PERSON} subject. Certain phrases are very common across all our sample discussions and while these are important for comprehension, they do not make adequate points. \textit{object understand, realize, debate, speak, stand, refer, explain, support, feel} are just some examples of `actions' that participants write about that do not generate points suitable for use in summaries.

    \subsection{Patterns}
      Common phrases used in discussion create another common class of poor quality points. Phrases such as \textit{`I'm correct [in saying]', `Something happened [to/that]', `[opposition] make the claim'} were common in all debates. They could not be used to add useful content to the summary and are removed. Rather than looking to define these with a general rule, point patterns are only added to this blacklist when they occur often enough.
  \section{Person Nominal Subject Amalgamation}
