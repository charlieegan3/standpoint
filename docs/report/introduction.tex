\chapter{Introduction\label{chap:introduction}}
  More people are online than ever before. Comment threads and discussion forums allow us to spend spare time participating as contributors - rather than just consuming content. From the latest blockbuster title to yesterday's celebrity misdemeanor, there's an online conversation already well underway.

  These discussions, where statements are encoded in natural language, represent a large untapped resource of ideas. A higher-level view of this information would be useful and interesting to many.

  This project explores an approach for summarization of such information. At the core of the approach is the notion of a `point' - a short refined argument statement. We model a user's argument in a discussion as consisting of one or more points. We then use these points as the content units when summarizing the discussion. Points are extracted from text and clustered to give a summary of the discussion. To test our implementation's performance we ran an evaluation using summaries generated from various political debates sourced from online discussions \cite{walker2012corpus}.

  \section{Motivation}
    Summarization has been a long-running task in the field of Natural Language Processing. Summarization sub-tasks such as extraction and compression, where text is selected and removed to arrive at a summary, have become commonplace due to the complexities introduced by abstractive methods. Extractive methods have become largely statistical using Naive-Bayes \cite{kupiec1995trainable} approaches and, more recently, even Neural Networks \cite{svore2007enhancing}.

    Argumentation Mining, a newer area of study, has the aim of identifying argumentative discourse structure in text. Argumentation Mining has been successfully used in the processing of formal texts (where arguments are often stated more explicitly) such as parliamentary records \cite{palau2009argumentation} and legal documents \cite{montemagni2010semantic}. However, Argumentation Mining has also more recently been applied to more informal texts \cite{park2015conditional}. Such applications of Argument Mining on informal texts, coupled with summarization, encapsulates much of the novelty for this project.

    The idea behind using points was adopted from a previous project in the department that used a similar concept in a system with a focus on stance classification. While this implemented point identification, the tool was incapable of extracting points beyond subjects and objects and still used sentence extracts. The tool was capable of linking contrasting points, but only based on the presence of negation terms.

    This project was based on the concept of a point as well as the idea that informal argumentative discourse could be used to build a high-level summary of an online discussion. News article comment sections; forum threads; film \& product reviews and even extended email conversations are all candidate applications for such analysis.

  \section{Objectives}
    Starting with the definition for a point: a verb and it's context, we set a number of objectives for the project. Continuing the work of a past project \textbf{(cite Angrosh?)} that implemented the idea of a simple point, we wanted to build on this and improve on the extraction of points from text. Fundamentally, this means extracting more of the context for a verb than the subject and object.

    Another goal was to investigate relationships between points such as contrastive or co-occurring points --- this involved expanding the ways in which related points could be matched, beyond negation. As secondary objective, we wanted to investigate co-occurring points, points that were commonly raised by the same user in a discussion.

    The task of stance classification was also discussed. Another objective was to investigate if certain points were representative for a known stance. Stance was annotated on some of the corpus debates. Finally we set out to present this information in a way that was easy to interpret, this evolved into our discussion summaries.

    The project's success is evaluated with respect to these objectives.
