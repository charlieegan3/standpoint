\chapter{Introduction\label{chap:introduction}}
  More people are online than ever before. Online comment threads and forums allow us to participate as contributors - rather than just consumers - in our spare time. From the latest blockbuster title to yesterday's celebrity misdemeanor there's an online conversation already well underway.

  These discussions, where statements are encoded in natural language, represent a large untapped resource of ideas and opinions. A higher level view of this living dataset would be of interest in a number of fields ranging from politics \& marketing to research.

  This project explores an approach for summarization of such information. At the core of the approach is the notion of a `point' - a short refined statement. Points are extracted from text and grouped into summaries. We test our implementation's performance by running an evaluation using testing summaries generated for various social media debates.

  \section{Motivation}
    todo
  \section{Objectives}
    The core objectives of the project were to investigate and implement the following functionality:
    \begin{itemize}
      \item{A point extraction program that accounts for the subcategorisation of verbs.}
      \item{A program capable of linking of contrastive points.}
      \item{An interface to study points made about different topics by participants in online discussions.}
    \end{itemize}
    We also set the following secondary goals to be investigated should there have been the opportunity.
    \begin{itemize}
      \item{Implement a feature capable of linking points representative of a particular stance.}
      \item{Implement a means of linking points and the related supporting points in a discussion.}
    \end{itemize}
