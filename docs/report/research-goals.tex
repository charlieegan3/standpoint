\chapter{Research Goals\label{chap:res-goals}}
  Having given an overview of the related work and background to the project, we now go on to outline the project's goals.

  First we wanted to develop a robust means of identifying and extracting points from text that would serve as a foundation for later analysis. We needed to investigate not only if points were a viable content unit but if they could be better at succinctly presenting information than extracted sentences.

  We had to first develop a robust means of extracting points from text. Given plain text from a discussion, we would need 1) a pattern or signature that could be used to link points --- regardless of their exact phrasing --- and 2) a shorter extract that could be used to present the point to readers.

  We were also interested to build on extracted points by linking point pairs to model the discussion. Matching points and counter points using negation or antonyms was the first goal. Additionally, linking co-occurring points as a more generalized view on participant stances was also of interest.

  A means of presenting and testing the results of this analysis was also a requirement. This led us to the task of using points in summarization. We wanted to use point metadata extracted from text such as referenced topics, source post and negation to be used in formulating a structured summary that is useful to readers.

  Aware that the foundation analysis of extracting points has utility beyond that of summarizing discussion, we wanted to build our tools for analysis in a modular manner such that components may be reused in the future for further work.

  \textcolor{red}{(I feel like this section is missing something but I don't know how to add more detail without going beyond the scope of the project or into the details of the method)}
