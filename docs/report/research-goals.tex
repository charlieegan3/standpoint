\chapter{Research Goals\label{chap:res-goals}}
  Having given an overview of the related work and background to the project, we now go on to outline the project's goals.

  Building on points extraction, we wanted to use points as a new content unit in summarizing large discussions. Points allow sentences to be broken down into a number of atomic units that be compared and matched to other points. We wanted to use points and the analysis they enable to build a picture of the discussion.

  This first requires a robust means of identifying and extracting points from text that would serve as a foundation for later analysis. We needed to investigate not only if points were a viable content unit but if they could be better at succinctly presenting information than extracted sentences. Given plain text from a discussion, we would need 1) a pattern or signature that could be used to link points --- regardless of their exact phrasing --- and 2) a shorter extract that could be used to present the point to readers.

  We were also interested to build on these extracted points by linking pairs of points to model the discussion. Matching points and counter points using negation or antonyms was the first goal. Additionally, linking co-occurring points as a more generalized view on participant stances was also of interest. We wanted to use these relations between points as well as point metadata extracted from text such as referenced topics, source post and negation in formulating a structured summary that is useful to readers.

  Aware that the foundation analysis of extracting points has utility beyond that of summarizing discussion, we wanted to build our tools for analysis in a modular manner such that components may be reused in the future for further work.

  \textcolor{red}{(I feel like this section is missing something but I don't know how to add more detail without going beyond the scope of the project or into the details of the method)}
