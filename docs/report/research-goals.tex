\chapter{Research Goals\label{chap:res-goals}}
  Having justified the working on the topic, we now give an overview of our project's goals.

  First we wanted to develop a robust means of identifying and extracting points from text, this would serve as a foundation for later analysis. We wanted investigate not only if points were a viable content unit but if they could be better at succinctly presenting information than extracted sentences.

  We were also interested to expand on points by matching points to counter points using negation and antonyms. Co-occurring points were also of interest as a more general view on stances held by conversation participants. We also needed a way to present and test these. This led us to our final goal of using points as a content unit for a summarization task. Additionally, could metadata about points extracted from text such as referenced topics, source and negation and be used to formulate a structured summary that is useful to readers.

  Aware that the foundation analysis of extracting points could have utility beyond that of summarizing discussion, we wanted to build our tools for analysis in a modular manner such that components may be reused in the future for further work.

  \textcolor{red}{(I feel like this section is missing something but I don't know how to add more detail without going beyond the scope of the project or into the details of the method)}
