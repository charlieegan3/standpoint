\chapter{Research Goals\label{chap:res-goals}}
  Having given an overview of the related work and background to the project, we now go on to outline the project's goals.

  Building on points extraction, we wanted to use extracted points as a new type of content unit in summarizing large discussions. Points allow sentences to be broken down into a number of atomic units that can be compared and matched to other points. We wanted to use points, and the analysis they enable, to build a picture of the discussion.

  This first required a robust means of identifying and extracting points from text to serve as a foundation for later analysis. We needed to investigate not only whether points were a viable content unit, but if they could succinctly present information better than extracted sentences. Given plain text from a discussion, we would need 1) a pattern or signature that could be used to link points --- regardless of their exact phrasing --- and 2) a shorter extract that could be used to present the point to readers.

  We were also interested to build on these extracted points by linking them in different ways to model the discussion. Firstly, matching points and counter points using negation and antonyms was the main comparison of interest. Additionally, linking co-occurring points as a more generalized view on participant stances was also a goal. We wanted to use these relations between points; as well as point metadata extracted from text such as referenced topics, source post and negation; in formulating a structured summary that is useful to readers.

  We also needed to evaluate the results of the tool as a whole. Testing the value of our analysis, and whether the results were useful, was also a goal for the project. We wanted to show that our approach to points extraction was suitable for the summarization of discussions. Aware that the base analysis of extracting points has utility beyond that of summarizing discussion, we wanted to build our tools for analysis in a modular manner such that components may be reused in the future for further work on other tasks.
